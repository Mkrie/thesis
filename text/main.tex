\documentclass[a4paper,12pt]{article}

\usepackage{cmap}
\usepackage{mathtext}
\usepackage[T2A]{fontenc}
\usepackage[utf8]{inputenc}
\usepackage[english,russian]{babel}
\usepackage{indentfirst}
\frenchspacing


\usepackage{amsmath,amsfonts,amssymb,amsthm,mathtools} % AMS
\usepackage{icomma}

\DeclareMathOperator{\sgn}{\mathop{sgn}}

\newcommand*{\hm}[1]{#1\nobreak\discretionary{}
{\hbox{$\mathsurround=0pt #1$}}{}}

\usepackage{graphicx}
\graphicspath{{images/}{images2/}}
\setlength\fboxsep{3pt}
\setlength\fboxrule{1pt}
\usepackage{wrapfig}

\usepackage{array,tabularx,tabulary,booktabs}
\usepackage{longtable}
\usepackage{multirow}

\theoremstyle{plain}
\newtheorem{theorem}{Теорема}[section]
\newtheorem{proposition}[theorem]{Утверждение}
 
\theoremstyle{definition}
\newtheorem{corollary}{Следствие}[theorem]
\newtheorem{problem}{Задача}[section]
 
\theoremstyle{remark}
\newtheorem*{nonum}{Решение}

\usepackage{etoolbox}


\usepackage{extsizes}
\usepackage{geometry}
	\geometry{top=25mm}
	\geometry{bottom=35mm}
	\geometry{left=35mm}
	\geometry{right=20mm}

\usepackage{setspace}

\usepackage{lastpage}

\usepackage{soul}

\usepackage{hyperref}
\usepackage[usenames,dvipsnames,svgnames,table,rgb]{xcolor}
\hypersetup{
    unicode=true,
    pdftitle={Заголовок},
    pdfauthor={Автор},
    pdfsubject={Тема},
    pdfcreator={Создатель},
    pdfproducer={Производитель},
    pdfkeywords={keyword1} {key2} {key3},
    colorlinks=true,
    linkcolor=red,
    citecolor=black,
    filecolor=magenta,
    urlcolor=cyan
}

\usepackage{csquotes}

\usepackage{multicol}

\usepackage{tikz}
\usepackage{pgfplots}
\usepackage{pgfplotstable}

\newcommand{\nl}{\\ \indent}

%\author{}
\title{Измерительно-вычислительная система для оценки вертикального профиля двуокиси азота в атмосфере}
\date{18.02.23}

\begin{document}
\maketitle
\tableofcontents
\newpage
\section{Введение}
in process
\section{Основная часть}
in process
\subsection{DOAS}
in process
\subsection{Обратные задачи}
Рассмотрим неизвестный сигнал 
$f \in \mathcal{L}^2(\mathbb{R}^d)$ и предположим, 
что последний проверяется каким-либо сенсорным устройством, 
что приводит к вектору данных 
$\textbf{y} = [y_1, \ldots, y_L]$ из $L$ измерения. 
Восстановление $f$ из вектора данных $\textbf{y}$ 
называется обратной задачей.
\nl
Следующие допущения являются стандартными:
\begin{itemize}
\item Чтобы учесть неточности датчиков предполагается, 
что вектор данных $\textbf{u}$ это результат 
случайного вектора 
$\textbf{Y} = [Y_1, \ldots , Y_L] =
\left[ 
\langle f, \phi_1 \rangle,
\ldots ,
\langle f, \phi_L \rangle
\right]$, 
колеблющийся в соответствии с некоторым распределением шума. 
Записи $\mathbb{E}[\textbf{Y}] = \widetilde{\textbf{y}}$ 
называются идеальными измерениями(измерения, которые были бы получены в отсутствие шума).
\item Измерения предполагаются несмещенными и линейными,
т.е.
$\mathbb{E}[\textbf{Y}] = \Phi^{*}f$, 
для некоторых функционалов выборки 
$\{ \phi_1, \ldots , \phi_L \} 
\subset \mathcal{L}^2(\mathbb{R}^d)$ 
моделирующих систему сбора данных.
\end{itemize}
\subsection{Некорректность обратных задач}
Для решения обратной задачи можно аппроксимировать 
среднее $\mathbb{E}[Y]$ по его одновыборочной эмпирической 
оценке $\textbf{y}$ и решить линейную задачу:
\begin{equation}
\label{eq_1}
\textbf{y} = \textbf{G} \alpha
\end{equation}
К сожалению, такие задачи в целом некорректны:
\begin{itemize}
\item \textbf{Решений может не быть}. 
Если $\textbf{G}$ не сюръективен,
$\mathcal{R}(\textbf{G}) \subseteq \mathbb{R}^L$. 
Следовательно, зашумленный вектор данных $\textbf{y}$ 
не гарантируется принадлежность к $\mathcal{R}(\mathbb{G})$.
\item \textbf{Может существовать более одного решения}. 
Если $L < N$ действительно (или, в более общем смысле, если 
$\textbf{G}$ не инъективен), 
$\mathcal{N}(\textbf{G}) \neq \{ \textbf{0} \}$. 
Следовательно, если $\alpha^{*}$ является решением 
\eqref{eq_1}, тогда $\alpha^{*} + \beta$ также является решением
$\forall \beta \in \mathcal{N}(\textbf{G})$:
\[
\textbf{G}(\alpha^{*} + \beta) =
\textbf{G} \alpha^{*} + \textbf{G} \beta =
\textbf{G} \alpha^{*} = \textbf{y}
\]
\item \textbf{Решения могут быть численно неустойчивыми}.
Если $\textbf{G}$ например, сюръективно, то
$\textbf{G}^{\dagger} = 
\textbf{G}^{T}(\textbf{G} \textbf{G}^{T})^{-1}$
является правой инверсией $\textbf{G}$,
а также 
$\alpha^{*}(\textbf{y}) =
\textbf{G}^{T}(\textbf{G} \textbf{G}^{T})^{-1}$
является решением \eqref{eq_1}. 
У нас есть тогда
\[
||\alpha^{*}(\textbf{y})|| \leq
||\textbf{G}||_{2}
||(\textbf{G}^{T} \textbf{G})^{-1}||_2
||\textbf{y}||_2 =
\underbrace{
\dfrac{
\sqrt{\lambda_{\text{max}}(\textbf{G}^{T}\textbf{G}})
}{
\lambda_{\text{min}}(\textbf{G}^{T}\textbf{G})
}}_{\text{может быть очень большим}} 
||\textbf{y}||_2, \qquad 
\forall \textbf{y} \in \mathbb{R}^L.
\]
\end{itemize}
\section{Постановка задачи}
В экспериментальных исследованиях результатом 
измерения является конечный набор значений
$\pmb{\xi} = (\xi_1, \ldots, \xi_N)$,
где $\xi_i$ -- наклонное содержание
$\text{NO}_2$ для зинитного угла направления
визирования $z_i$.
По набору этих значений требуется оценить значения вертикального
распределения $\text{NO}_2$ $n(h)$ в заданном наборе точек
области определения.
\nl
В задаче восстановления $\text{NO}_2$ результат измерения
имеет вид:
\begin{equation}
\xi_i = 
\int_{0}^{H}m(z_i, h)n(h)dh + \nu_i =
\sum\limits_{j=1}^{M}m(z_i, h_j) n(h_j) \triangle h
+ \nu_i = (a_i, n) + \nu_i
\end{equation}
где $a_i$ -- послойная воздушная масса,
которая рассчитывается из модели переноса излучения.
Она связывает наклонное содержание
$\text{NO}_2$ $\pmb{\xi}$
с вертикальным распределением концентраций $n(h)$.
\nl
Схему измерений можно представить в виде:
\begin{equation}
\pmb\xi = A \pmb{n} + \pmb{\nu}
\end{equation}
На основании измерения вектора
$\pmb\xi = (\xi_1, \ldots, \xi_N)$
требуется оценить значение вектора
$Un \in \mathbb{R}^N$, 
где $U: \mathbb{R}^N \rightarrow \mathbb{R}^N$ --
оператор, которому сопоставляется матрица с учётом 
априорной информации о значениях координат вектора 
$\pmb{n} = (n_1, \ldots, n_M)$.
\begin{description}
\item[1)]
Координаты вектора $\pmb n$ неотрицательны:
$n_i > 0, i=1, \ldots, M.$
\item[2)]
Восстановленный профиль $\pmb n$ является унимодальным:
\begin{equation}
\begin{cases}
n_i \leq n_{i+1}, \quad i \leq k \\
n_{i+1} \leq n_i, \quad i \geq k
\end{cases}, \qquad
1 \leq k \leq M, \qquad
i = 1, \ldots, M-1.
\end{equation}
\end{description}
\section{Предложенный метод}
Решение будем искать как решение задачи на минимакс,
тогда:
\begin{equation}
\min_{\pmb n} \max_{i = 1, \ldots, M}
\left| \xi_i - (a_i, \pmb f)\right|
\end{equation}
С учетом априорной информации:
\begin{equation}
\min_{\pmb f, k} \max_{i = 1, \ldots, M}
\left|
\xi_i - (a_i, \pmb f)
\right|
\end{equation}
Данную задачу можно представить как задачу линейного программирования:
\[
(c, d) \sim \min
\]
\[
c = (1, 0, \ldots, 0), \qquad
d = (z, n_1, \ldots, n_M), \qquad
1 \leq k \leq M
\]
\[
\begin{cases}
n_1-n_2 \leq 0 \\
n_2-n_3 \leq 0 \\
\qquad \ldots \\
n_{k-1} - n_k \leq 0
\end{cases}
\quad 
\begin{cases}
n_N-n_{N-1} \leq 0 \\
n_{N-1}-n_{N-2} \leq 0 \\
\qquad \ldots \\
n_{k+1} - n_{k} \leq 0
\end{cases}
\quad
\begin{cases}
-n_1 \leq 0 \\
-n_2 \leq 0 \\
\quad \ldots \\
-n_N \leq 0
\end{cases}
\]
\[
\begin{cases}
z_1 \geq
\left|
\sum_{j=1}^N a_{1j}f_j - \xi_1
\right| \\
z_2 \geq
\left|
\sum_{j=1}^N a_{2j}f_j - \xi_2
\right| \\
\qquad \ldots \\
z_n \geq
\left|
\sum_{j=1}^N a_{nj}f_j - \xi_n
\right|
\end{cases}
\]

\section{Эксперименты}
in process
\section{Заключение}
in process

\newpage
\addcontentsline{toc}{section}{Список используемой литературы}
\begin{thebibliography}{}
    \bibitem{litlink1}  Author 1  -  "Name 1"
    \bibitem{litlink2}  Author 2  -  "Name 2"
    \bibitem{other-link-name}  Author 3  -  "Name 3"
\end{thebibliography}
\end{document}




